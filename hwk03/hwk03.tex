% Another example tex file
% Oberlin College
% Isaac Hollander McCreery

\documentclass[11pt]{amsart}

% document information
\newcommand{\hwk}{3}
\newcommand{\honor}{We affirm that we have adhered to the honor code on this assignment.}

% Another example tex file
% Oberlin College
% Isaac Hollander McCreery

% document information
\newcommand{\course}{CSCI 364}
\newcommand{\semester}{Spring 2013}
\newcommand{\name}{Artificial Intelligence}
\newcommand{\student}{Charles Marks, Isaac Hollander McCreery, Nick Towbin-Jones}
\newcommand{\lstudent}{Marks, McCreery, Towbin-Jones}

% packages
\usepackage[parfill]{parskip} % begin paragraphs with an empty line rather than an indent
\usepackage{fullpage} % use more of the page
\usepackage{amsmath,amssymb,amsthm} % math stuff
\usepackage{paralist}
\usepackage{algorithm}
\usepackage[noend]{algorithmic}
\usepackage{fancyhdr} % headers

% formatting pages, etc.
\renewcommand{\baselinestretch}{1.1}
\setlength{\headheight}{15.2pt}
\setlength{\headsep}{15pt}
\setlength{\textheight}{592pt}
\setlength{\footskip}{30pt}
\renewcommand{\headrulewidth}{0pt}
\renewcommand{\footrulewidth}{0pt}

% format the header
\fancyhf{}
\lhead{\lstudent}
\rhead{\today}
\cfoot{\thepage}
\pagestyle{fancy}

% format the title
\renewcommand\maketitle{
\begin{center}
	\textsc{\Large \name: Homework \#\hwk} \\ [12pt]
	\student \\ [12pt]
	\today
\end{center}
}

% list formatting
%\setcounter{secnumdepth}{0}
\renewcommand{\theenumi}{\alph{enumi}.}
\renewcommand{\labelenumi}{\theenumi}

%%
% The following are definitions so that you can use numbered lemmas, claims, etc.
%%
\newtheorem{lemma}{Lemma}
\newtheorem*{lem}{Lemma}
\newtheorem*{thm}{Theorem}
\newtheorem{definition}{Definition}
\newtheorem{notation}{Notation}
\newtheorem*{claim}{Claim}
\newtheorem*{fclaim}{False Claim}
\newtheorem{observation}{Observation}
\newtheorem{conjecture}[lemma]{Conjecture}
\newtheorem{theorem}[lemma]{Theorem}
\newtheorem{corollary}[lemma]{Corollary}
\newtheorem{proposition}[lemma]{Proposition}
\newtheorem*{rt}{Running Time}

\newcommand{\es}{\textsc{es} }


\begin{document}
\maketitle

\section*{Part 1: Probability Conundrum}

\section*{Part 2: Bayes' Theorem}

\section*{Part 3: Bayesian Networks}

\begin{enumerate}

\item
\begin{align*}
P(A \wedge B \wedge C \wedge D \wedge E)
	&= P(A) P(B) P(C) P(D | A \wedge B) P(E | B \wedge C) \\
	&= 0.2 \times 0.5 \times 0.8 \times 0.1 \times 0.3 \\
	&= 0.0024
\end{align*}

\item
\begin{align*}
P(\neg A \wedge \neg B \wedge \neg C \wedge \neg D \wedge \neg E)
	&= P(\neg A) P(\neg B) P(\neg C) P(\neg D | \neg A \wedge \neg B) P(\neg E | \neg B \wedge \neg C) \\
	&= 0.8 \times 0.5 \times 0.2 \times 0.9 \times 0.7 \\
	&= 0.0504
\end{align*}

\item
\begin{align*}
P(\neg A \wedge B \wedge C \wedge D \wedge E)
	&= P(\neg A) P(B) P(C) P(D | \neg A \wedge B) P(E | B \wedge C) \\
	&= 0.8 \times 0.5 \times 0.8 \times 0.6 \times 0.3 \\
	&= 0.0576
\end{align*}

\end{enumerate}

\section*{Part 5: AI in the World}

\begin{description}

\item[Bayesian belief networks for adaptive management] Hans Jørgen Henriksen and Heidi Christiansen Barlebo cite that
Bayesian networks (BNs) are often helpful in highly complex problems where "there is a scarcity and uncertainty in the
data used in making the decision and the factors are interlinked" (1026).  BNs allow stakeholders to work out with
experts what causal links exist in a system in order to better develop models for possible outcomes.  In their work
with pesticide reduction instruments\----contracts with farmers\----and their impact on groundwater safety, the authors
found that "rather costly compensations" would be required to ensure a 95\% probability that the water resources would
remain safe (1029).  When a special uncertainty arose where stakeholders disagreed, an additional "perception of
vulnerability" state was added to address this disagreement rather than favor one group over another (1029).  In their
follow-up, the authors found that two water managers involved in the process thought that BNs provided an alternative to
standard welfare economics, as BNs "could help to delineate the complexities and also handle some of the uncertainties"
associated with water management (1031).

Source:
Henriksen, Hans Jørgen \& Heidi Christiansen Barlebo.
"Reflections on the use of Bayesian belief networks for adaptive management".
\emph{Journal of Environmental Management}.
88 (2008) 1025-1036.

\item[Other application]

\end{description}

\section*{Part 6: Project Prep}

\begin{description}

\item[Bayesian belief networks with the Oberlin Project's Green Arts District]
The Oberlin Project's proposed Green Arts District on the block just East of Tappan Square in Oberlin has been both
lauded and criticized for its innovation and impact.  Constructing a Bayesian belief network (BN) modeling the
construction of the District and its various impacts might provide a clearer vision of the inputs and results of such a
project.  In Henriksen and Barlebo's work above, the authors produced a BN through an involved stakeholder process to
allow stakeholders to agree upon the interrelations in a complex policy system, providing a precedent we might follow:
\renewcommand{\theenumi}{(\roman{enumi})}
\renewcommand{\labelenumi}{\theenumi}
\begin{inparaenum}
\item define the context;
\item identify factors, actions and indicators;
\item build pilot networks;
\item collect data;
\item define states;
\item construct CPTs; and
\item collect feedback from stakeholders.
\end{inparaenum}
\renewcommand{\theenumi}{\alph{enumi}.}
\renewcommand{\labelenumi}{\theenumi}
The results might be, among other data, predictions for the economic impact of the project\----income brought to
Oberlin, gentrification and changes real estate prices\----environmental impact of the project\----energy saved,
materials used\----and social impact\----recognition, leadership and proof-of-concept value, changes in tensions between
the Town and the College.

\item[Other project]

\end{description}

\honor

\end{document}
