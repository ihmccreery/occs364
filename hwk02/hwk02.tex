% Another example tex file
% Oberlin College
% Isaac Hollander McCreery

\documentclass[11pt]{amsart}

% document information
\newcommand{\hwk}{2}
\newcommand{\honor}{We affirm that we have adhered to the honor code on this assignment.}

% Another example tex file
% Oberlin College
% Isaac Hollander McCreery

% document information
\newcommand{\course}{CSCI 364}
\newcommand{\semester}{Spring 2013}
\newcommand{\name}{Artificial Intelligence}
\newcommand{\student}{Charles Marks, Isaac Hollander McCreery, Nick Towbin-Jones}
\newcommand{\lstudent}{Marks, McCreery, Towbin-Jones}

% packages
\usepackage[parfill]{parskip} % begin paragraphs with an empty line rather than an indent
\usepackage{fullpage} % use more of the page
\usepackage{amsmath,amssymb,amsthm} % math stuff
\usepackage{paralist}
\usepackage{algorithm}
\usepackage[noend]{algorithmic}
\usepackage{fancyhdr} % headers

% formatting pages, etc.
\renewcommand{\baselinestretch}{1.1}
\setlength{\headheight}{15.2pt}
\setlength{\headsep}{15pt}
\setlength{\textheight}{592pt}
\setlength{\footskip}{30pt}
\renewcommand{\headrulewidth}{0pt}
\renewcommand{\footrulewidth}{0pt}

% format the header
\fancyhf{}
\lhead{\lstudent}
\rhead{\today}
\cfoot{\thepage}
\pagestyle{fancy}

% format the title
\renewcommand\maketitle{
\begin{center}
	\textsc{\Large \name: Homework \#\hwk} \\ [12pt]
	\student \\ [12pt]
	\today
\end{center}
}

% list formatting
%\setcounter{secnumdepth}{0}
\renewcommand{\theenumi}{\alph{enumi}.}
\renewcommand{\labelenumi}{\theenumi}

%%
% The following are definitions so that you can use numbered lemmas, claims, etc.
%%
\newtheorem{lemma}{Lemma}
\newtheorem*{lem}{Lemma}
\newtheorem*{thm}{Theorem}
\newtheorem{definition}{Definition}
\newtheorem{notation}{Notation}
\newtheorem*{claim}{Claim}
\newtheorem*{fclaim}{False Claim}
\newtheorem{observation}{Observation}
\newtheorem{conjecture}[lemma]{Conjecture}
\newtheorem{theorem}[lemma]{Theorem}
\newtheorem{corollary}[lemma]{Corollary}
\newtheorem{proposition}[lemma]{Proposition}
\newtheorem*{rt}{Running Time}

\newcommand{\es}{\textsc{es} }


\begin{document}
\maketitle

\section*{Part 1: Alpha-Beta Search}

\section*{Part 2: Minimax}

\begin{enumerate}

\item
If \es can identify a winning move at the top of the game tree, \es would
never have a reason for making a different move from the one identified by minimax.  A \emph{winning
move} for \es is a move such that, no matter how the opponent responds, \es 
has another winning move, or has a move that leads directly to a win.  Put another way, taking a
succession of repeated winning moves guarantees \es a win.

\item
Assuming that the best possible move found by \es based on a complete minimax search
would lead to a draw, \es might want to choose a different move than the one identified
by minimax.

Since the opponent can make errors, it is possible that there is some move that would lead to the
opponent not choosing the optimal move, which might allow \es a winning move.  More
concretely, consider two moves:
\begin{enumerate}
\item a move where all responses by the opponent lead to a draw;
\item a move where the optimal response by the opponent leads to a loss for \textsc{es}, but all the
moves recommended by the opponent's heuristic function lead to winning moves for \textsc{es}.
\end{enumerate}
Clearly \es should take (b) because \es will then win.

\end{enumerate}

\section*{Part 3: CSPs}

\section*{Part 4: Propositional Logic}

\subsection*{Robbery and a Salt}

\begin{enumerate}
\item
\item
\item
\end{enumerate}

\subsection*{Multiple Choice}

\begin{enumerate}
\item
\item
\item
\end{enumerate}

\section*{Part 5: AI in the World}

\section*{Part 6: Project Prep}

\subsection*{Go}
Can we build an AI player that can beat Ike or other (somewhat novice) players in a game of go?  Go
remains an unsolved game, though there are certainly AI algorithms around that are strong enough to
beat even advanced players.  An implimentation of minimax would be at the core of such a player, as
there are times in the game\----the end game\----when looking ahead is extremely important, but the
bulk of the work would be in heuristics, involving research into heuristic development, both general
and specifically for go, and likely extensive testing.  Go requires, for most human players, a keen
sense of feeling rather than calculation, and attempting to encode that feeling would be a complex,
subtle, and edifying process.

\honor

\end{document}
